\documentclass{article}
\usepackage[utf8]{inputenc}
\usepackage{enumerate}
\usepackage{csquotes}

\title{%
  A Critical Analysis of Epistemic Luck \\
  \large Process Reliabilism and The Truth Tracking Theory }

\author{Ousmane Amadou | 1003013222}
\date{PHL232H1 December 2017, Evan Taylor}

\begin{document}


\maketitle

% Section 0: Introduction, Background and Thesis
As of December 2017 no proposed theories of knowledge have managed
to specifiy knowledge conditions that are immune to the problem of
epistemic luck. % SOURCE: online The most common approach of specifiying conditions for
someones knowing a proposition, is to modify the \textbf{Classical Analysis of Knowledge}:
\begin{displayquote} % SOURCE: Gettier is knowledge JTB
   S knows that P if and only if \\
  (i) P is true. \\
  (ii) S beleives that P, and \\
  (iii) S is justified in beleiving that P.
\end{displayquote} so that each condition is necessary for knowledge, and jointly
sufficent for knowledge. Epistemic Luck is a generic term ascribed to instances of
knowledge, where a belief in a proposition P is true merely by luck. In other words,
epistemic luck that can be used to describe any number of ways in which it can be
accidental, coincidental, or fortuitous that a person has a true belief. To understand why epistemic luck has
proved to be a hard problem in epistemology, consider the following example(s) of knowledge claims
that suffer from epistemic luck:

\begin{displayquote}
\textbf{Example 1.}
\end{displayquote}
The focus of this paper is to compare and contrast the strengths of the process reliabist and
truth tracking theories of knowledge when it comes to solving the problem
of epistemic luck. I will argue that, if justification is required for
knowledge then Process Reliabilism should be preffered over the Truth Tracking
Theory as a general solution to epistemic luck.

% Section 1: Formulating The Prefer Relation
To begin I will first give a provisional formulation of what it could mean to
prfer Process Reliabilism over the Truth Tracking Theory as a general solution to epistemic luck.
For two theories of knowldge $T_{1}$ and $T_{2}$, consider the following
formulation preferedness:
\begin{displayquote}
    \textbf{Defintion 1.} As a general solution to Epistemic Luck,
    \textit{$T_{1}$ should be prefferred over $T_{2}$ as a general solution
    to epistemic luck} if and only if at least two of the following conditions are met:
    \begin{enumerate}
        \item[(i)] $T_{1}$ is a more plausible account of knowledge than $T_{2}$.
        \item[(ii)] $T_{1}$ resolves a larger number of epistemic luck knowledge claims than $T_{2}$.
        \item[(iii)] $T_{1}$ offers the favorable basis for further investigation of epistemic luck.
    \end{enumerate}
\end{displayquote}

% Section 2: Premise 1 | Argument for Process Reliabilisms Plausibility
A
% Section 3: Premise 2 | Argument for Truth Trackings Resolution Count
% Section 4: Premise 3 | Argument for Favorable Basis

% Section 5: Conclusion and Final Remarks
\end{document}
