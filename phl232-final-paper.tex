\documentclass{article}
\usepackage[utf8]{inputenc}
\usepackage{enumerate}
\usepackage{csquotes}

\title{%
  A Critical Analysis of Epistemic Luck \\
  \large Process Reliabilism and The Truth Tracking Theory }

\author{Ousmane Amadou | 1003013222}
\date{PHL232H1 December 2017, Evan Taylor}

\begin{document}


\maketitle

% Section 0: Introduction, Background and Thesis
As of December 2017 no proposed theories of knowledge have managed
to specifiy knowledge conditions that are immune to the problem of
epistemic luck. % SOURCE: online
The most common approach of specifiying conditions for someones knowing
a proposition, is to provide an account of knowledge that is a modification
of the \textbf{Classical Analysis of Knowledge}:
\begin{displayquote} % SOURCE: Gettier is knowledge JTB
   S knows that P if and only if \\
  (i) P is true. \\
  (ii) S beleives that P, and \\
  (iii) S is justified in beleiving that P.
\end{displayquote} so that each condition is necessary for knowledge, and jointly
sufficent for knowledge. Among these attempts include \textbf{Robert Nozick's Truth Tracking Theory}:
\begin{displayquote}
  \textit{S knows that P} if and only if
  \begin{enumerate}
      \item[(i)] P is true.
      \item[(ii)] S beleives that P.
      \item[(iii)] If P were false, S would not beleive that P.
      \item[(iv)] If P were true, S would beleive that P.
  \end{enumerate}
\end{displayquote}
and  \textbf{Process Reliabilism} which asserts that:
\begin{displayquote}
  \textit{S knows that P} if and only if
  \begin{enumerate}
      \item[(i)] P is true.
      \item[(ii)] S beleives that P.
      \item[(iii)] S is justified in beleiving that P.
      \item[(iv)] S is only justified in beleiving that P, if that beleif
      was formed using a reliable beleif forming process.
  \end{enumerate}
\end{displayquote}
The focus of this paper is to compare and contrast the strengths of the process
reliabist and truth tracking theories of knowledge when it comes to solving the
problem of epistemic luck. I will argue that the Truth Tracking theory should be
preffered over the Truth Tracking Theory as a general solution to epistemic luck by
describing Epistemic Luck, then providing a characterization of what it could
mean to prefer one theory over another as a genral solution to Epistemic Luck, and
showing that Truth Tracking Theory satisfies this characterization.

Epistemic Luck is a generic term ascribed to instances of knowledge, where a belief in a proposition P is true merely by luck. In other words,
epistemic luck can be used to describe any number of ways in which it can be
accidental, coincidental, or fortuitous that a person has a true belief. To understand
why epistemic luck has proved to be a hard problem in epistemology, consider
the following example(s) of knowledge claims that suffer from epistemic luck:

\begin{displayquote}
\textbf{Example 1.}
\end{displayquote}


%%% Section 1: Formulating The Prefer Relation
To begin I will first give a provisional defintion of what it could mean to
prefer the Truth Tracking Theory over Process Reliabilism as a general solution to
epistemic luck. For two theories of knowldge $T_{1}$ and $T_{2}$, consider the
following formulation preferedness:
\begin{displayquote}
    \textbf{Defintion 1.} As a general solution to Epistemic Luck,
    \textit{$T_{1}$ should be prefferred over $T_{2}$ as a general solution
    to epistemic luck} if and only if at least two of the following conditions are met:
    \begin{enumerate}
        \item[(i)] $T_{1}$ is a more tenable account of knowledge than $T_{2}$.
        \item[(ii)] $T_{1}$ resolves a larger number of epistemic luck knowledge claims than $T_{2}$.
        \item[(iii)] $T_{1}$ offers the favorable basis for further investigation of epistemic luck.
    \end{enumerate}
\end{displayquote}
  There may be other combinations of conditions that could accurrately capture
what it means to prefer Truth Tracking over Process Reliabilism as a general
solution to epistemic luck. In fact, I am not claiming that the conditions
specified give the most accurate definition preferedness. However, I do think
this defintion suffices as a basis for compxaring these two theories because
the most important aspects of these theories are being accounted for and weighted
equally.
%%% Section 2: Premise 1 | Argument for Process Reliabilisms Tenability
% Part 1: Introduce Premise

% Part 2(a): Rejection of Closure Principle hurts TT's tenability

One of the biggest criticsms of the Truth-Tracking theory is that it requires
the rejection of the highly intuitive closure principle. That is, according
to the truth tracking theory if S knows P and S  knows that  P $\Rightarrow$ Q,
then S does not necessarily know that Q. Nozick, however, claims that rejecting
the closure princple is actually a feature of the theory as a response to this
criticsm. He maintains that the fact the truth tracking theory illumaniting a
certain property of knowledge. Indeed, there is no substantial evidence that would
prove other wise. Even after considering that, it is still more plausible
to prefer a theory that coheres with our logical beleif system. Abonding the
closure princple hurts the Truth Tracking's theory's tenablity. Another criticsm
of the Truth-Tracking theory is that it is vague. For instance, how are we
supposed to interpret the Subjunctive conditional? In particular, how exactly
do we know in scenerios where it were the case that P? One proposed semantics, is
the possible worlds semantics that says:
\begin{displayquote}
    If $\phi$ were true, then $\psi$ is true if and only if:
  \begin{enumerate}
    \item[(i)] $\psi$ is true at some $\phi$-world w
    (where a $\psi$-world is just one in which $\psi$ is true); and
    \item[(ii)] w is closer to actuality than is any $\phi$ and $\neg \psi$
    world w'.
  \end{enumerate}
\end{displayquote}

However, if one is not moved by any such discrimination requirement,
one will not be moved by this objection. Therefore Process Reliabilism
is the more tenable account of knowledge.

% Part 3(a): The Truth Tracking theory is too vague
%% Subjunctive Conditional, Possible Worlds (What counts as a possible world)

% Part 3(b): Response to 3(a) | Process Reliabilism is vague as well!
% Process Reliabilism is not without it's issues either.
% Firstly, what is meant by process?
% Secondly, the generality problem?

%%% Section 3: Premise 2 | Argument for Truth Trackings Resolution Count
Suppose $E$ is the set of all knowedge-claims that are true merely by luck.
It is conceivable to partition this set into different classes, or types
of epistemically lucky knowldge claims. That is $E = C_{1} \cup C_{2} \cup
C_{3} ... \cup C_{i}$ where each $C_{j}, \forall j <= i$ represents a class
of epistemic knowledge claims. It is important to note that it is perferctly
possible for an epistemically lucky knowledge claim to be of more than one type.
Furthermore, partioning epistemically lucky knowledge claims is a somewhat
aggergious task. The manner in which you could

Types:
* False Grounds
* Original Counter Examples
* General Formula
* Unclassifed

%%% Section 4: Premise 3 | Argument for Favorable Basis
An advantage of Process Reliabilism is that it is a theory of both
justification and knowledge. If correct, Process Reliabilism would
allow us to understand how knowledge is contstructed and would
offer a basis for analyzing epistemic luck itself. % Justify this
However, it is still debated whether knowledge actually requires justification.
If it were the case that knowledge did require justification, Process Reliabilism
would be valuable. However, if knowledge doesn't require justification
Process Reliabilism doesn't offer much value for further investigation
since it is fundamentally flawed in assuming knowledge require justification.
A response to this could reject the Process Relibilist theory of justification
In Goldmans What is JTB, we learn that a reliable beleif forming Process
is neither sufficient nor necessary for knowledge.

% Process Reliabilism is neither suffificent nor necessary for justification
% (use lottery paradox)
%%% Section 5: Conclusion and Final Remarks

\end{document}
