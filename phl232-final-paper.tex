\documentclass{article}
\usepackage[utf8]{inputenc}
\usepackage{enumerate}

\title{%
  A Critical Analysis of Epistemic Luck \\
  \large Process Reliabilism and The Truth Tracking Theory }

\author{Ousmane Amadou | 1003013222}
\date{PHL232H1 December 2017, Evan Taylor}

\begin{document}


\maketitle

% Section 0: Introduction, Background and Thesis
As of December 2017 no proposed theories of knowledge have managed to specifiy knowledge conditions that are immune to the problem of epistemic luck. To understand why epistemic luck has proved to be a hard problem in epistemology, we will first need to define epistemic luck and some relevant related concepts. The focus of this paper is to compare and contrast the strengths of the process reliabist and truth tracking theories of knowledge when it comes to solving the problem of epistemic luck. I will argue that, if justification is required for knowledge then Process Reliabilism should be preffered over y as a general solution to epistemic luck.

% Section 1: Formulating The Prefer Relation
Suppose $T_{prf}(x, y)$ is a binary relation defined as follows, for two theories of knowldge x and y;
\begin{center}
    x should be prefferred over y as a general solution to epistemic luck if at least two of the following conditions are met:
    \begin{enumerate}
        \item[i] x is a more plausible account of knowledge than y 
        \item[ii] x resolves a larger number of epistemic luck knowledge claims than y
        \item[iii] x offers the favorable basis for further investigation of epistemic luck
    \end{enumerate}
\end{center}

% Section 2: Premise 1 | Argument for Process Reliabilisms Plausibility
% Section 3: Premise 2 | Argument for Truth Trackings Resolution Count
% Section 4: Premise 3 | Argument for Favorable Basis

% Section 5: Conclusion and Final Remarks
\end{document}
