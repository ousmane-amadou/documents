\documentclass{article}
\usepackage{fullpage,amsmath,amssymb}
\usepackage{hyperref}
\title{CSC236H, Fall 2017\\
Assignment 2\\
Due November 12th, 10:00 p.m.}
\renewcommand{\today}{~}
\hypersetup{pdfpagemode=Fullscreen,
  colorlinks=true,
  linkfileprefix={}}
  
\newcommand{\floor}[1]{\lfloor #1 \rfloor}
  
  
\begin{document}
\maketitle
\vspace{-1.5cm}
\begin{itemize}
	\item You may work in groups of no more than \textbf{three} students, and you should 
	produce a single solution in a PDF file named \texttt{a2.pdf}, submitted to {MarkUs}. Submissions must be \textbf{typed}. 
	\item Please refer to the course information sheet for the \textbf{late submission policy}.
\end{itemize}
\vspace{1\baselineskip}

\begin{enumerate}


\item Let $k\in\mathbb N$ and $m\in\mathbb R$. 
Consider the following function $S:\mathbb R\times\mathbb N\to \mathbb R$. 
$$S(m,k) = \left\{
\begin{array}{ll}
3m & \mbox{if } k = 0\\
4S(\frac{m}{2}, k-1) + 3m & \mbox{if } k \geq 1
\end{array} \right .$$
Find a closed-form expression for $S(m,k)$ using repeated substitution.\\
Prove the correctness of the expression you obtained.\\
\\\\
Assume $k > 0$
\begin{align*}
 S(m, k) & = 4S(\frac{m}{2}, k-1) +3m 
&\text{ After $i = 1$ substitution}\\
& = 4(4S(\frac{m}{4},k-2) + \frac{3m}{2}) +3m\\
& = {4}^{2}S(\frac{m}{{2}^{2}},k-2) + ({2}^{2-1} + 1)3m
&\text{ After $i = 2$ substitution}\\
& = {4}^{2}(4S(\frac{m}{{2}^{3}},k-3) + \frac{3m}{4}) + ({2}^{2-1} + 1)3m\\
& = {4}^{3}S(\frac{m}{{2}^{3}},k-3) + ({2}^{3-1} + {2}^{2-1} + 1)3m
&\text{ After $i = 3$ substitution}\\
&\quad\quad\quad\quad\quad\quad\quad\quad \vdots\\
& = {4}^{k}S(\frac{m}{{2}^{k}}, k - k) + ({2}^{k-1} + {2}^{k-2} \dotsi + 1)3m
&\text{ After $i = k$ substitution}\\
& = {4}^{k}\frac{3m}{{2}^{k}} + ({2}^{k-1} + {2}^{k-2} \dotsi + 1)3m\\
& = {2}^{k}3m + ({2}^{k-1} + {2}^{k-2} \dotsi + 1)3m\\
& = ({2}^{k} + {2}^{k-1} \dotsi + 1)3m\\
& = (\frac{1 - {2}^{k+1}}{1-2})3m= ({2}^{k+1} - 1)3m
\end{align*}
\newpage
Let's define the predicate, $P(k)$, the following way:\\
$S(m, k) = ({2}^{k+1} - 1)3m$ for a given $m \in\mathbb{R}$ and $k \in \mathbb{N}$\\
Now let $m \in\mathbb{R}$\\
\textbf{Base Case:}\\
Let $k = 0$
\\$S(m, k) = 3m = ({2}^{0+1}-1)3m$
\\Therefore $P(0)$ holds\\
\textbf{Inductive Step:}\\
Let $k > 0, k \in\mathbb{N}$ and $j \in\mathbb{N}$ such that $0\le j \le k$\\
Assume $P(j)$ holds for all $j$ [IH]\\
$S(m, k + 1) = 4S(\frac{m}{2}, k) + 3m $\\
$S(\frac{m}{2}, k) = ({2}^{k} - 1)\frac{3m}{2}$ by the induction hypothesis and because $\frac{m}{2} \in\mathbb{R}$ as $m \in\mathbb{R}$\\
$S(m, k + 1) =  4({2}^{k} - 1)\frac{3m}{2} + 3m = ({2}^{k+1} - 2)3m + 3m$\\
$S(m, k + 1) = ({2}^{k+1} - 2 + 1)3m = ({2}^{k+1} - 1)3m$\\
Therefore $P(k+1)$ holds.\\
Thus, by the principle of strong induction, $P(k)$ holds for all $k \in\mathbb{N}$

\item Consider the following algorithm:\\

\textbf{Precondition:} $A$ is a non-empty list of elements.\\
\textbf{Postcondition:} returns the reverse of $A$.\\
\begin{tabbing}
	MM\=MM\=MM\=MM\=MM\=\kill
	\>\textbf{def} $reverse\_list(A)$:\\
	1.\>  \> i = 1  \\ 
	2.\>  \> t = len(A)\\ 
	3.\>  \> B = [\:]  \\
	4.\>  \> \textbf{while} (i $<$ len(A) + 1):\\       
	5.\>  \> \> B.append(A[t - i])\\       
	6.\>  \> \> i = i + 1\\
	7.\>  \> \textbf{return} B
\end{tabbing}

\begin{enumerate}
	\item Give an appropriate loop invariant for the purpose of proving partial correctness for the above program 
	with respect to its given specification. \\
	Prove your loop invariant.\\\\
	Let A be a non empty list of elements.\\
	Let's define the loop invariant, $LI(k)$, as follows:\\
	If the loop iterates at least $k$ times, then
	\begin{enumerate}
	   \item ${i}_{k} = 1 + k$ and $1 \le {i}_{k}\le$ len(A) + 1
	   \item B is the last $k$ elements of A in reverse order of how they appear in A
	\end{enumerate}
\newpage
    Proof of $LI(k):$\\\\
	\textbf{Base Case:}\\
	Let $k = 0$\\
	Then by line 1, ${i}_{0} = 1$\\
	Also by line 3, B = [\:]\\
	${i}_{0} = 1 + 0$ and $1 \le {i}_{0} \le$ len(A) + 1 as A is non empty\\
	B is the last 0 elements of A in reversed order, as B is empty\\
	Therefore $LI(0)$ holds\\
	\textbf{Inductive Step:}\\
	Let $n \in\mathbb{N}$ and assume $LI(n)$ holds. [IH]\\
	Then ${i}_{n} = n + 1$\\
	Then A[t - ${i}_{n}$] = A[t - n - 1]\\
	Then A[t - ${i}_{n}$] is the ${n+1}^{th}$ last element of A.\\
	Before line 5, B has n elements by $LI(n)$\\
	By line 5, the ${n+1}^{th}$ first element of B is the ${n+1}^{th}$ last element of A\\
	Also the first n elements of B is the last n elements of A in reversed order by $LI(n)$\\
	Then B is the last $n+1$ elements of A in reversed order\\
	By line 6, ${i}_{n+1} = {i}_{n} + 1 = 1 + n + 1$\\
	Also $1 \le {i}_{n} < {i}_{n+1}$ by $LI(n)$\\
	Since, after n iterations, another iteration exists, ${i}_{n} <$ len(A) + 1\\
	Then ${i}_{n+1} \leq$ len(A) + 1 \\
	Then $1 \le {i}_{n+1} \le $ len(A) + 1\\
	Therefore $LI(n+1)$ holds.
	\item Use your loop invariant from part (a) to prove partial correctness for the above program 
	with respect to its given specification.\\\\
	Let A be a non empty list of elements.\\
	Assume the loop terminates after $k$ iterations.\\
	Then $\neg$(${i}_{k} <$ len(A) + 1)\\
	Then ${i}_{k} \ge$ len(A) + 1\\
	Also, ${i}_{k} \le$ len(A) + 1, by $LI(k)$ part i\\
	Then ${i}_{k} =$ len(A) + 1\\
	Then $k=$ len(A) by $LI(k)$ part i\\
	Also, B is the last $k$ elements of A in reversed order by $LI(k)$ part ii\\
	Then B is the len(A) elements of A in reversed order\\
	Then B is the entire list A in reversed order\\
	The program returns B\\
	So the program returns the correct value. 
	
	
\end{enumerate}

\newpage
\item Consider the following recursive algorithm:\\
\textbf{Preconditions:} $A$ is an array of integers. $f$ and $\ell$ are integers such that $0\leq f \leq \ell < len(A)$.\\
\textbf{Postconditions:} For all integers $b$, if $b$ occurs more than $\frac{len(A[f:\ell +1])}{2}$ times in $A[f:\ell +1]$, 
then $maj(A,f,\ell)$ returns $b$. 

\begin{tabbing}
	MM\=MM\=MM\=MM\=MM\=\kill
	\>\textbf{def} $maj(A,f, \ell)$:\\
	1.\>  \>  \textbf{if} $f ==\ell$:\\ 
  2.\>  \>  \>  \textbf{return} $A[f]$\label{base}\\
	3.\>  \>  $m = \lfloor (f+\ell)/ 2\rfloor$\label{mdef}\\
	4.\>  \>  $x = maj(A,f,m)$\label{majrec1}\\
	5.\>  \>  $c = 0$\\
	6.\>  \>  $i = f$\label{winit}\\
	7.\>  \>  \textbf{while} $i \leq \ell$:\label{wloop}\\
	8.\>  \> \> \textbf{if} $A[i] = x$: \\ 
	9.\>  \> \>  \> $c = c +1$\label{comps}\\
	10.\>  \> \>  $i = i +1$\label{endloop}\\
  11.\>  \>  \textbf{if} $c >   (\ell -f+1)/2$:\label{test}\\
  12.\>  \>  \>  \textbf{return} $x$\label{majret1}\\
  13.\>  \>  \textbf{return}  $maj(A,m+1,\ell)$\label{majrec2}
\end{tabbing}


\begin{enumerate}
\item Let $T(n)$ denote the worst-case running time of the algorithm below on inputs of size $n$.
Write a \textbf{recurrence relation} satisfied by $T$. 
Then, give an asymptotic \textbf{upper-bound} for the worst-case running time of the algorithm. \\

Make sure to define $n$ precisely and justify that your recurrence is correct 
by referring to the algorithm to describe how you obtained each term in your answer.\\
For this part of the question (and only this part), you may assume that  $len(A[f:\ell+1])$ and $len(A)$ are powers of 2.\\

Let $n = len(A[f:\ell+1]) = \ell+1 - f$\\\\
When $n = 1$, we have $\ell = f$ by rearranging $1 = \ell + 1 = f$.
Thus the condition on line 1 is fulfilled and line 2 is run.
Since random access is constant in a list, $T(1)$ takes constant time,
call it $a$.\\\\

If $n$ were to be larger than 1 then lines 1-13 would execute. Line 3 is a calculation which takes constant time, call it $b$.\\

Note that since $n > 1$, by assumption $len(A[f:\ell+1])$ is a power of 2, and $n = len(A[f:\ell+1])$, $n$ is an even number. Rearranging and adding terms to $n = \ell + 1 - f$, we get $n - 1 + 2f = \ell + f$. Since $n - 1$ is odd, and $2f$ is even, $\ell + f$ is odd and thus $m = \lfloor (f+\ell)/ 2\rfloor = (\ell + f - 1)/2 (*)$.

Line 4 is a recursive call with input size $len(A[f:m+1]) = m + 1 - f = (\ell + f - 1)/2 + 1 - f = n/2$. Thus line 4 takes T($n/2$) time.\\\\

Lines 5-6 are basic operations and take constant time. call it $g$.\\\\

Lines 8-10  take constant time, call it $d$. They make up the loop body of the loop from lines 7 - 10. $i$ starts as having the value of $f$ by line 6, increments by 1 after each iteration by line 10, and stops when it is greater than $\ell$ by line 7. Thus lines 8-10 repeat for $\ell - f + 1 = n$ times. Lines 8-10 takes $nd$ time.

Since we are measuring worst case, we will assume the condition on line 11 fails and line 13 runs.

Line 13 is a recursive call with input size $len(A[m+1: \ell+1]) = \ell + 1 -(m+1) = \ell - (\ell + f - 1)/2 = n/2$. Thus line 13 takes T($n/2$) time.\\\\

So when $n > 1, T(n) = b + T(n/2) + g + nd + T(n/2) = 2T(n/2) + nd + k$ if we let $k = g + b$.

Thus the final equation is \\

 \[T(n) = \begin{cases} 
a & n = 1 \\
2T(n/2) + nd + k & n >1 \\
\end{cases}
\]

Since $log_2(2) = 1$ and $nd + k \in \Theta(n)$ we can apply the master theorem from class and say $T(n) \in O(nlogn)$.
\item State an \textbf{invariant} for the loop on line 7 that describes the value of $c$ at the end of every iteration of the loop.\\
\textbf{Prove} your loop invariant.
 Define $LI(k)$ as:\\
 If the loop iterates at least $k$ times then
 \begin{enumerate}
 	\item $f \leq i_k \leq \ell + 1$
 	\item $c_k$ is the number of times $x$ appears in $A[f:i_k]$\\
 \end{enumerate}

Proof of $LI(k)$:\\\\
Base case, proving $LI(0)$:\\
$i_0 = f$ 				\quad\quad (by line 6)\\
$ \leq \ell + 1$		\quad\quad (since $f \leq \ell$ by precondition)\\
Thus $f \leq i_0 \leq \ell + 1$.\\
$A[f:f]$ is the empty list, so the number of times anything is in it is 0.
$c_k = 0$ by line 5, so it is the correct amount of times $x$ appears in $A[f:f]$.\\
Thus base case holds.\\\\
Inductive step, assume $LI(n)$ for some $n \in \mathbb{N}$ [IH]\\
WTP: $LI(n + 1)$\\
Note that since there was $n + 1$ iterations, $i_n \leq \ell$. Thus
$i_n + 1 \leq \ell + 1$. By line 10, $i_n + 1 = i_{n+1}$. Thus $i_{n+1} \leq \ell + 1$. By IH $i_n \geq f$. Thus $i_{n+1} \geq f$.\\
Thus $ f \leq i_{n+1} \leq \ell + 1$\\
Note that $A[f:i_{n+1}]$ is just $A[f:i_n]$ with $A[i]$ appended to the end.
By IH, $c_n$ counts the number of times $x$ appears in $A[f:i_n]$. If $A[i] = x$ then there are $c_n + 1$ occurrences of $x$. $A[i] = x$ is the condition on line 8. Line 9 executes and $c_{n+1} = c_n +1$. Thus if  $A[i] = x$, $c_{n+1}$ is the correct value. If $A[i]\neq x$ then there are $c_n$ occurrences of $x$. $A[i] \neq x$ means the condition on line 8 fails and $c_n$ doesn't change. Thus $c_{n+1} = c_n$.
Thus $c_{n+1}$ is the correct value. In both cases $c_{n+1}$ returns the number of occurences of $x$ in $A[f:i_{n+1}]$.\\

Thus $LI(k)$ is proven.
  

\item Prove the \textbf{correctness} of $maj$ with respect to the given specification.\\
Use induction for the correctness proof and follow the steps (i)-(iv) below.\\

You may assume that the loop on line 7 terminates without proof.\\
Also, you may use the following fact without proof:\\
If $u \leq v$, $v+1 \leq w$, and $b$ occurs more than $s+s'$
times in $B[u:w+1]$, then\\
either
$b$ occurs more than $s$ times in $B[u:v+1]$ or
$b$ occurs more than $s'$
times in $B[v+1:w+1]$. (Recall that $B[i:j+1]$ is the slice of $B$ from $i$ to $j$)


%\begin{spacing}
%\vspace*{-0.2in}
%\end{spacing}

\begin{enumerate}
\item Define a predicate $P(k)$ that denotes the correctness of $maj$ for an input with size $k$.\\\\
%such that proving $\forall k \in \ints^+.P(k)$ proves the claim.
Define $P(k)$ as:\\
If $A$ is an array of integers and $f, l$ are integers such that $0\le f \le l < len(A)$ and $len(A[f:\ell +1]) = k$ for $k \in \mathbb{{Z}^{+}}$, then $maj(A, f, l)$ terminates and returns $b \in \mathbb{{Z}^{+}}$ such that $b$ occurs more than $\frac{len(A[f:\ell +1])}{2}$ times in $A[f:\ell +1]$
\item Prove the base case.\\\\
WTP: $P(1)$\\\\
Let $A$ be an array of integers and $f, l$ are integers such that $0\le f \le l < len(A)$\\
Since $len(A[f:\ell + 1]) = 1$, the only element in $A[f:\ell +1]$ is $A[f]$.
Let $b$ be an arbitrary integer such $b$ occurs more than $1/2$ times in  $len(A[f:\ell + 1])$. The only way for this to be possible is if $b = A[f]$.\\\\
Note that $len(A[f:\ell + 1]) = 1$ means  $\ell + 1 -f = 1$ which means $\ell = f$.
Thus the condition on line 1 passes, and line 2 runs. Thus the program terminates and $A[f] = b$ is returned by line 2.\\\\
Thus base case $P(1)$ holds.
\item State the induction hypothesis.\\\\
Let $n$ be a positive integer. Assume for all positive integers $j$ such that $1 \leq j < n,P(j)$ holds [IH]\\\\\\

\item Prove the induction step.\\\\
WTP: $P(n)$\\
Before we do this, let us prove some useful facts.
\quad\begin{enumerate}
	\item[](*): $len(A[f:m+1])/2 + len(A[m+1:\ell + 1])/2 = len(A[f:\ell + 1]/2)$\\
	
	Proof: There are two cases. \\\\
	Case: $len(A[f:\ell + 1]$ is even\\
	$len(A[f:\ell + 1] = \ell + 1 - f \Rightarrow $\\
	$len(A[f:\ell + 1] + 2f - 1 = \ell + f \Rightarrow $\\
	$\ell +f$ is odd since $len(A[f:\ell + 1] + 2f$ is even and 1 is odd $\Rightarrow$\\
	$\lfloor(\ell+f)/2\rfloor = (\ell+f-1)/2 \Rightarrow$\\
	$m = (\ell + f-1)/2$\\
	Thus $len(A[f:m+1])/2 + len(A[m+1:\ell + 1])/2 = $ \\
	$(m+1-f)/2 + (\ell+1 - (m+1))/2 = $\\
	$\frac{\frac{\ell + f -1}{2} +1 -f}{2} + \frac{\ell - \frac{\ell + f -1}{2}}{2} =$\\
	$\frac{\ell -f + 1}{4} + \frac{\ell -f + 1}{4} = $\\
	$\frac{\ell -f+1}{2} = len(A[f:\ell +1])/2$\\
	First case works\\\\
    Case: $len(A[f:\ell + 1]$ is odd\\
    $len(A[f:\ell + 1] = \ell + 1 - f \Rightarrow $\\
    $len(A[f:\ell + 1] + 2f - 1 = \ell + f \Rightarrow $\\
    $\ell +f$ is even since $len(A[f:\ell + 1] + 2f$ is odd and 1 is odd $\Rightarrow$\\
    $\lfloor(\ell+f)/2\rfloor = (\ell+f)/2 \Rightarrow$\\
    $m = (\ell + f)/2$\\
    Thus $len(A[f:m+1])/2 + len(A[m+1:\ell + 1])/2 = $ \\
    $(m+1-f)/2 + (\ell+1 - (m+1))/2 = $\\
    $\frac{\frac{\ell + f}{2} +1 -f}{2} + \frac{\ell - \frac{\ell + f}{2}}{2} =$\\
    $\frac{\ell -f + 2}{4} + \frac{\ell -f}{4} = $\\
    $\frac{\ell -f+1}{2} = len(A[f:\ell +1])/2$\\
    Thus both cases work and we are done proving (*)\\\\
	\item[](**):After the loop on line 7 terminates(happens by assumption), $c$ is the number of times $x$ occurs in $A[f:\ell + 1]$\\
	
	Proof:\\\\
	By assumption, the loop terminates, so by line 7, $i_k \geq \ell + 1$, where $k$ is number of times the loop executed up to and including the last iteration. By the first part of Loop invariant in question 3b, $i_k \leq \ell + 1$.
	Thus $i_k = \ell + 1$. By the second part of Loop invariant in question 3b, $c_k$ then is the number of occurences of $x$ in $A[f:\ell + 1]$. Since $c_k$ is the value of $c$ after the last iteration, The value $c$ after the loop terminates is the number of occurrences of $x$ in $A[f:\ell +1]$. The proof of (**) is done.\\\\
\end{enumerate}

Now we can prove $P(n)$\\
Proof:\\\\
Let $A$ be an array of integers and $f, l$ are integers such that $0\le f \le l < len(A)$\\
Let $b$ be an integer that appears more than $len(A[f:\ell+1])/2$ times in $A[f:\ell +1]$ such that $len(A[f:\ell +1]) = n$. 

Since $f \leq \ell \Rightarrow f + f \leq  f + \ell$, which means $f \leq (f+\ell)/2$.
Since $f$ is an integer, $f \leq  \lfloor(f+\ell)/2\rfloor = m$ and thus $f \leq m$.

Since we assumed $n >1$, $\ell > f \Rightarrow \ell + \ell > f + \ell$ and thus, $\ell >  (f + \ell)/2$. Since $m =  \lfloor(f+\ell)/2\rfloor \leq  (f+\ell)/2$, we have $\ell > m$. Since $\ell$ is an integer, we know $\ell \geq m + 1$.



By (*) we know that $len(A[f:m+1])/2 + len(A[m+1:\ell + 1])/2 = len(A[f:\ell + 1]/2$. 

Combining (*) with the fact that  $\ell \geq m + 1$ and $f \leq m$ and $f \leq \ell$, we can apply the assumption given in the question and say that either $b$ appears more than $len(A[f:m+1])/2$ times in $A[f:m+1]$ or that $b$ appears more than $len(A[m+1:\ell + 1])/2$ times in $A[m+1:\ell + 1]$. We'll split the proof into two cases.

\quad\begin{enumerate}
	\item[]Case $b$ appears more than $len(A[f:m+1])/2$ times in $A[f:m+1]$\\\\
	Note that $len(A[f:m+1]) = m + 1 - f \leq \ell - f < \ell + 1 - f = len(A[f:\ell+1]) = n $. Thus we can apply IH on $len(A[f:m+1])$.\\
 By IH we can conclude that  $maj(A,f,m)$ terminates and returns the correct value. Since  $b$ appears more than $len(A[f:m+1])/2$ times in $A[f:m+1]$ in this case, $b$ is returned by $maj(A,f,m)$. By line 5, value of $x$ is $maj(A,f,m) = b$. By (**), after the loop on line 7 terminates. $c$ is the number of occurences of $x$ in $A[f:\ell +1]$. Since $b$ is defined to appear more than $len(A[f:\ell+1]) = \ell -f +1$ times, the condition on line 11 is fulfilled and $x$ is returned by line 12. Since $x=b$, the function terminates and correctly returns $b$.\\
 
 \item[]Case $b$ appears more than $len(A[m+1:\ell+1)]/2$ times in $A[m+1:\ell +1]$.\\
 Note that no other value besides $b$ occurs more than $len(A[f:\ell +1])/2$ times in $A[f:\ell+1]$. Suppose there's another value $w$ that occurs more than $len(A[f:\ell +1])/2$ times in $A[f:\ell+1]$. Then the total occurrences of both $w$ and $b$ would be greater than $ len(A[f:\ell +1])/2 + len(A[f:\ell +1])/2 = len(A[f:\ell +1])$ which is a contradiction.\\
 Note that $len(A[m+1:\ell+1) = \ell + 1 -(m + 1)= \ell - m \leq \ell - f < \ell + 1 - f = len(A[f:\ell+1]) = n $. Thus we can apply IH on $len(A[m+1:\ell+1])$.\\
 By IH we can conclude that  $maj(A,m+1,\ell)$ terminates and  returns the correct value. Since  $b$ appears more than $len(A[m+1:\ell)]/2$ times in $A[m+1:\ell +1]$ in this case, $b$ is returned by $maj(A,m+1,\ell)$. Assuming that $x \neq b$, line 12 will fail to run since $x$ will not occur more thab $len(A[f:\ell +1])/2$ times in $A[f:\ell+1]$. Then line 13 will run and will return $maj(A,m+1,\ell)$ which is $b$.
\end{enumerate}
Thus in both cases, the program terminates and correctly returns $b$. Our inductive step and overall proof is done and $maj(A,f,\ell)$ returns the correct value and terminates.
\end{enumerate}
\end{enumerate}
\end{enumerate}

\end{document}
