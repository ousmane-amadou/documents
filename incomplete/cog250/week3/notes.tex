\documentclass[twoside]{article}
\setlength{\oddsidemargin}{0.25 in}
\setlength{\evensidemargin}{-0.25 in}
\setlength{\topmargin}{-0.6 in}
\setlength{\textwidth}{6.5 in}
\setlength{\textheight}{8.5 in}
\setlength{\headsep}{0.75 in}
\setlength{\parindent}{0 in}
\setlength{\parskip}{0.1 in}

%
% ADD PACKAGES here:
%

\usepackage{amsmath,amsfonts,amssymb,graphicx,mathtools,flexisym}

%
% The following commands set up the lecnum (lecture number)
% counter and make various numbering schemes work relative
% to the lecture number.
%
\newcounter{lecnum}
\renewcommand{\thepage}{\thelecnum-\arabic{page}}
\renewcommand{\thesection}{\thelecnum.\arabic{section}}
\renewcommand{\theequation}{\thelecnum.\arabic{equation}}
\renewcommand{\thefigure}{\thelecnum.\arabic{figure}}
\renewcommand{\thetable}{\thelecnum.\arabic{table}}

%
% The following macro is used to generate the header.
%
\newcommand{\lecture}[4]{
   \pagestyle{myheadings}
   \thispagestyle{plain}
   \newpage
   \setcounter{lecnum}{#1}
   \setcounter{page}{1}
   \noindent
   \begin{center}
   \framebox{
      \vbox{\vspace{2mm}
    \hbox to 6.28in { {\bf COG250H1: Introduction to Cognitive Science
	\hfill Fall 2018} }
       \vspace{4mm}
       \hbox to 6.28in { {\Large \hfill Lecture #1: #2  \hfill} }
       \vspace{2mm}
       \hbox to 6.28in { {\it Lecturer: #3 \hfill Scribes: #4} }
      \vspace{2mm}}
   }
   \end{center}
   \markboth{Lecture #1: #2}{Lecture #1: #2}

   {\bf Note}: {\it LaTeX template courtesy of UC Berkeley EECS dept.}

   {\bf Disclaimer}: {\it These notes have not been subjected to the
   usual scrutiny reserved for formal publications.  They may be distributed
   outside this class only with the permission of the Instructor.}
   \vspace*{4mm}
}
%
% Convention for citations is authors' initials followed by the year.
% For example, to cite a paper by Leighton and Maggs you would type
% \cite{LM89}, and to cite a paper by Strassen you would type \cite{S69}.
% (To avoid bibliography problems, for now we redefine the \cite command.)
% Also commands that create a suitable format for the reference list.
\renewcommand{\cite}[1]{[#1]}
\def\beginrefs{\begin{list}%
        {[\arabic{equation}]}{\usecounter{equation}
         \setlength{\leftmargin}{2.0truecm}\setlength{\labelsep}{0.4truecm}%
         \setlength{\labelwidth}{1.6truecm}}}
\def\endrefs{\end{list}}
\def\bibentry#1{\item[\hbox{[#1]}]}

%Use this command for a figure; it puts a figure in wherever you want it.
%usage: \fig{NUMBER}{SPACE-IN-INCHES}{CAPTION}
\newcommand{\fig}[3]{
			\vspace{#2}
			\begin{center}
			Figure \thelecnum.#1:~#3
			\end{center}
	}
% Use these for theorems, lemmas, proofs, etc.
\newtheorem{theorem}{Theorem}[lecnum]
\newtheorem{lemma}[theorem]{Lemma}
\newtheorem{proposition}[theorem]{Proposition}
\newtheorem{claim}[theorem]{Claim}
\newtheorem{corollary}[theorem]{Corollary}
\newtheorem{definition}[theorem]{Definition}
\newenvironment{proof}{{\bf Proof:}}{\hfill\rule{2mm}{2mm}}

% **** IF YOU WANT TO DEFINE ADDITIONAL MACROS FOR YOURSELF, PUT THEM HERE:

\newcommand\E{\mathbb{E}}

\begin{document}
%FILL IN THE RIGHT INFO.
%\lecture{**LECTURE-NUMBER**}{**DATE**}{**LECTURER**}{**SCRIBE**}
\lecture{3}{Categorization II (Classical Theory and Prototype Theory (pt.1))}{Anderson Todd}{Ousmane Amadou}
%\footnotetext{These notes are partially based on those of Nigel Mansell.}


\section{Classical Theory}
\subsection{Motivation}
Smith: Categorization is running on some feature of the world. Categorization
is in some sense a bottom up process.

Response: Categorization is  a top down process. It is more plausible that
we use concepts to categorize objects. We are imposing a concept into cateogries.

Aristotle championed the classical theory of concepts.

Somehow we are using concepts to do categorization.
\subsection{Theory}
The classical theory is organized around the idea that all instances of a category share a common  of properties.
A list of features that are individual necessary and collectively sufficient to represent the concept.

\textbf{Evidence: }

\textbf{Criticisms: }
1. Wittgenstein on the game

\textbf{Related Ideas: } Problem of Universals, Substance Theory, Object Oriented
Programming, Aristotle on Essence

\section{Prototype Theory}
\subsection{What is salience?}
Underlies prototype theory. Saliency differes among cultures and individuals by
universals/prototypes in prototype theory require generalizablity. Cannot formalize Tversky's formula because of salience
function.

\subsection{Theory}
Developed as an alternative to the Classical Theory Of Concepts

\textbf{Proponents: } Elanor Rosch

\textbf{Evidence: }

\textbf{Criticsms: } Falls prey to homuncular fallacy (what is salience)

\subsection{Geometric Notion - Tversky's Formula}
$$ Sim(I, J) = af(I, J) + bf(I, J) + cf(I, J) $$
where:
f is a function that measures the salience of each set of features.
a, b,  and  c are  parameters  that  determine the relative contribution
of the three feature sets.

https://rationalwiki.org/wiki/Prototype_theory

\end{document}
