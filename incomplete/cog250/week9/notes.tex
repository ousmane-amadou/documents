\documentclass[twoside]{article}
\setlength{\oddsidemargin}{0.25 in}
\setlength{\evensidemargin}{-0.25 in}
\setlength{\topmargin}{-0.6 in}
\setlength{\textwidth}{6.5 in}
\setlength{\textheight}{8.5 in}
\setlength{\headsep}{0.75 in}
\setlength{\parindent}{0 in}
\setlength{\parskip}{0.1 in}

%
% ADD PACKAGES here:
%

\usepackage{amsmath,amsfonts,amssymb,graphicx,mathtools,flexisym}

%
% The following commands set up the lecnum (lecture number)
% counter and make various numbering schemes work relative
% to the lecture number.
%
\newcounter{lecnum}
\renewcommand{\thepage}{\thelecnum-\arabic{page}}
\renewcommand{\thesection}{\thelecnum.\arabic{section}}
\renewcommand{\theequation}{\thelecnum.\arabic{equation}}
\renewcommand{\thefigure}{\thelecnum.\arabic{figure}}
\renewcommand{\thetable}{\thelecnum.\arabic{table}}

%
% The following macro is used to generate the header.
%
\newcommand{\lecture}[4]{
   \pagestyle{myheadings}
   \thispagestyle{plain}
   \newpage
   \setcounter{lecnum}{#1}
   \setcounter{page}{1}
   \noindent
   \begin{center}
   \framebox{
      \vbox{\vspace{2mm}
    \hbox to 6.28in { {\bf COG250H1: Introduction to Cognitive Science
	\hfill Fall 2018} }
       \vspace{4mm}
       \hbox to 6.28in { {\Large \hfill Lecture #1: #2  \hfill} }
       \vspace{2mm}
       \hbox to 6.28in { {\it Lecturer: #3 \hfill Scribes: #4} }
      \vspace{2mm}}
   }
   \end{center}
   \markboth{Lecture #1: #2}{Lecture #1: #2}

   {\bf Note}: {\it LaTeX template courtesy of UC Berkeley EECS dept.}

   {\bf Disclaimer}: {\it These notes have not been subjected to the
   usual scrutiny reserved for formal publications.  They may be distributed
   outside this class only with the permission of the Instructor.}
   \vspace*{4mm}
}
%
% Convention for citations is authors' initials followed by the year.
% For example, to cite a paper by Leighton and Maggs you would type
% \cite{LM89}, and to cite a paper by Strassen you would type \cite{S69}.
% (To avoid bibliography problems, for now we redefine the \cite command.)
% Also commands that create a suitable format for the reference list.
\renewcommand{\cite}[1]{[#1]}
\def\beginrefs{\begin{list}%
        {[\arabic{equation}]}{\usecounter{equation}
         \setlength{\leftmargin}{2.0truecm}\setlength{\labelsep}{0.4truecm}%
         \setlength{\labelwidth}{1.6truecm}}}
\def\endrefs{\end{list}}
\def\bibentry#1{\item[\hbox{[#1]}]}

%Use this command for a figure; it puts a figure in wherever you want it.
%usage: \fig{NUMBER}{SPACE-IN-INCHES}{CAPTION}
\newcommand{\fig}[3]{
			\vspace{#2}
			\begin{center}
			Figure \thelecnum.#1:~#3
			\end{center}
	}
% Use these for theorems, lemmas, proofs, etc.
\newtheorem{theorem}{Theorem}[lecnum]
\newtheorem{lemma}[theorem]{Lemma}
\newtheorem{proposition}[theorem]{Proposition}
\newtheorem{claim}[theorem]{Claim}
\newtheorem{corollary}[theorem]{Corollary}
\newtheorem{definition}[theorem]{Definition}
\newenvironment{proof}{{\bf Proof:}}{\hfill\rule{2mm}{2mm}}

% **** IF YOU WANT TO DEFINE ADDITIONAL MACROS FOR YOURSELF, PUT THEM HERE:

\newcommand\E{\mathbb{E}}

\begin{document}
%FILL IN THE RIGHT INFO.
%\lecture{**LECTURE-NUMBER**}{**DATE**}{**LECTURER**}{**SCRIBE**}
\lecture{9}{Problem Solving (pt.2)}{Anderson Todd}{Ousmane Amadou}
%\footnotetext{These notes are partially based on those of Nigel Mansell.}

% **** YOUR NOTES GO HERE:
\subsection{Problem Solving as Search}
\begin{itemize}
  \item Motivation: Given that problem spaces are extremey large. How are humans
        intelligently navigating this search space?
  \item Chess is a *wellformed problem*... alas most of the problems we encounter
        in the real world are *ill-formed problems*. These problems don't have
        explicit path constraints.
  \item When you consider the similarites between the two types of problems, you
        will begin to notice that *problem formulation place a big role*.
  \item The Mutliated Chess Board
  \begin{itemize}
    \item When people encounter this problem the immediate response is to try
          and mentally countevery chess tile. This strategy requires a LARGE amount of human competition.
    \item A reformulation of the problem in this case leads to a trivial solution.

  \end{itemize}

\end{itemize}

\subsection{The Interconnection Problem}
\begin{itemize}
  \item There are two schools of thought that are concerned with solving the
        Interconnection Problem.
  \item Gestalt School Of Thought
    \begin{itemize}
      \item They were terrible experimental designers
      \item They were doing their seminal work in the 30s, 40s and 50s... and
            they were german.. basically, no american touched gestalt pyschology
            becasue nazis
      \item They beleived that solving an insight problem is less like an action,
            and more of a facet of perception.. they posited that problem solving
            invovled perceptual **restructuring**
      \item Todd likes this school of thought because it is highly intuitive to
            him... Introspectively, he notices a restructuring of perception when
            he gains insight or has some sort of eureka moment
      \item Necker Cube
      \item Fixation is the whole shabang... its the problem
    \end{itemize}
\end{itemize}

\subsection{Insight Problem Solving}
\begin{itemize}
  \item Weisberig = the hero insight psychology needs.. and Alba
  \item The 9 dot problem is an insight problem. The 9 dot problem is a classic
        example of how problem formulation affects...
\end{itemize}

Key Concepts: Structural Reframing, Combinatorial Explosion
\end{document}
