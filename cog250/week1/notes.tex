\documentclass[twoside]{article}
\setlength{\oddsidemargin}{0.25 in}
\setlength{\evensidemargin}{-0.25 in}
\setlength{\topmargin}{-0.6 in}
\setlength{\textwidth}{6.5 in}
\setlength{\textheight}{8.5 in}
\setlength{\headsep}{0.75 in}
\setlength{\parindent}{0 in}
\setlength{\parskip}{0.1 in}

%
% ADD PACKAGES here:
%

\usepackage{amsmath,amsfonts,amssymb,graphicx,mathtools,flexisym}

%
% The following commands set up the lecnum (lecture number)
% counter and make various numbering schemes work relative
% to the lecture number.
%
\newcounter{lecnum}
\renewcommand{\thepage}{\thelecnum-\arabic{page}}
\renewcommand{\thesection}{\thelecnum.\arabic{section}}
\renewcommand{\theequation}{\thelecnum.\arabic{equation}}
\renewcommand{\thefigure}{\thelecnum.\arabic{figure}}
\renewcommand{\thetable}{\thelecnum.\arabic{table}}

%
% The following macro is used to generate the header.
%
\newcommand{\lecture}[4]{
   \pagestyle{myheadings}
   \thispagestyle{plain}
   \newpage
   \setcounter{lecnum}{#1}
   \setcounter{page}{1}
   \noindent
   \begin{center}
   \framebox{
      \vbox{\vspace{2mm}
    \hbox to 6.28in { {\bf COG250H1: Introduction to Cognitive Science
	\hfill Fall 2018} }
       \vspace{4mm}
       \hbox to 6.28in { {\Large \hfill Week #1: #2  \hfill} }
       \vspace{2mm}
       \hbox to 6.28in { {\it Lecturer: #3 \hfill Scribes: #4} }
      \vspace{2mm}}
   }
   \end{center}
   \markboth{Lecture #1: #2}{Lecture #1: #2}

   {\bf Note}: {\it LaTeX template courtesy of UC Berkeley EECS dept.}

   {\bf Disclaimer}: {\it These notes have not been subjected to the
   usual scrutiny reserved for formal publications.  They may be distributed
   outside this class only with the permission of the Instructor.}
   \vspace*{4mm}
}
%
% Convention for citations is authors' initials followed by the year.
% For example, to cite a paper by Leighton and Maggs you would type
% \cite{LM89}, and to cite a paper by Strassen you would type \cite{S69}.
% (To avoid bibliography problems, for now we redefine the \cite command.)
% Also commands that create a suitable format for the reference list.
\renewcommand{\cite}[1]{[#1]}
\def\beginrefs{\begin{list}%
        {[\arabic{equation}]}{\usecounter{equation}
         \setlength{\leftmargin}{2.0truecm}\setlength{\labelsep}{0.4truecm}%
         \setlength{\labelwidth}{1.6truecm}}}
\def\endrefs{\end{list}}
\def\bibentry#1{\item[\hbox{[#1]}]}

%Use this command for a figure; it puts a figure in wherever you want it.
%usage: \fig{NUMBER}{SPACE-IN-INCHES}{CAPTION}
\newcommand{\fig}[3]{
			\vspace{#2}
			\begin{center}
			Figure \thelecnum.#1:~#3
			\end{center}
	}
% Use these for theorems, lemmas, proofs, etc.
\newtheorem{theorem}{Theorem}[lecnum]
\newtheorem{lemma}[theorem]{Lemma}
\newtheorem{proposition}[theorem]{Proposition}
\newtheorem{claim}[theorem]{Claim}
\newtheorem{corollary}[theorem]{Corollary}
\newtheorem{definition}[theorem]{Definition}
\newenvironment{proof}{{\bf Proof:}}{\hfill\rule{2mm}{2mm}}

% **** IF YOU WANT TO DEFINE ADDITIONAL MACROS FOR YOURSELF, PUT THEM HERE:

\newcommand\E{\mathbb{E}}

\begin{document}
%FILL IN THE RIGHT INFO.
%\lecture{**LECTURE-NUMBER**}{**DATE**}{**LECTURER**}{**SCRIBE**}
\lecture{1}{Introduction}{Anderson Todd}{Ousmane Amadou}
%\footnotetext{These notes are partially based on those of Nigel Mansell.}

% **** YOUR NOTES GO HERE:
\section{Brief Introduction to Cognitve Science}
\subsection{Motivation}
Purpose: Set the stage for providing visions of cognitive sceince, introduce
sub disciplines, briefly describe key events in cognitive science.

Topic involving cognitive science:
\begin{itemize}
  \item AlphaGo
  \item Computer Vision
  \item High Frequency Trading
  \item Literacy + Grammar
\end{itemize}

In cognitive science the problem of the mind is being approached from multiple
angles. One of these approaches is characterized by employing \textbf{design thinking}
to create models of the mind. The brain as a machine is a \textbf{metaphor} that
you will often see in cognitive science. The fundamental belief is that if science
can figure out how to design and construct a mind, then science can figure out how the mind works.
This is one of the reasons why the study of artificial intellgience is so intimately
connected to the study of the mind.

There are, however, problems with this approach. Many machine learning systems these
days are very black box. That is, although we understand the basic principles
of how these systems work, they are not fully yet understood.

\subsection{Theoretical Constructs and Plausibility}

A theoretical construct (a.k.a hypothetical construct) is description of
an ideal object (an idea) that is not directly observable. Theoretical constructs
are judged on how accurately they describe the object.

Streams of evidence....
Converging input and elegant output -> Plausibility

You want your converging input and elegant
If your plausibility is deep then you may have a theory that is profound. This is
the goal of integrating multiple disciplines.

Example: Attachment Theory and Nueroscience in relation to trust in relationships...
What does trust mean in mathematical terms.

A theoretical construct is strong if and only if it is: multi apt, is supported by converging evidence, is
elegantly formulated.

More Writing Points: Explain Multi Aptnes, Describe the connection between aptness
and metaphor

\subsection{Metaphor}
The word "metaphor" is derived from the greek word metaphori meaning "transfer"
or to carry over. Metaphors in cognitive science prodvide a structural backbone
for knowledgetransfer between concepts, where knowledge transfer in this sense
describes a "carrying over of meaning" from one concept to another.

For example, the phrase "Sam is a pig!" is a metaphor that conveys the understanding T
that Sam is a sloppy or disgusting individual.


\subsection{What is Cognitive Science?}
Ironically, one thing science can't account for in scientific explanations is how
humans produce scientific explanations. Because of this there is a sense in which we are deeply
alienated from the world. That is, there is a \textbf{you-shaped hole} in
our understanding of it. (notice that you-shaped hole is a metaphor) One of the goals
of cognitive science to fill this gap, and complete our understanding of the natural world.

In general, cognitive science seeks to develop a common language for describing \textbf{cognitive phenomena}
that can be understood through multiple disciplines with its the core considered to be philosophy.

There are three 'visions' for cognitive science.

\textbf{Generic Nominalism} is a broad approach to cognitive science. Nominalism
in this context can be best understood as a universal name for a particular idea while
generic corresponds to a group or class of related ideas. Generic Nominalism
describes the idea that cognitive science is simply a name for the disciplines
involved in studying the mind. Thus the following discplines are referred to
as the cognitive sciences: Artificial Intelligence, Cognitive Psychology, Philosophy
of Mind, Cultural Anthropology, Neuroscience, Semiotics and Linguistics. In other words,
the only requirement for doing cognitive science for a generic nominalist is to
do something related to the study of the mind. This vision is generally not accepted
in third generation cognitive science.

\textit{ Related Ideas: Natural Philosophy (as a generic name for scientific inquiry),
         Artificial Intelligence (a generic name for a loosely defined family of technologies),
                    }

\textbf{Interdisciplinary Ecclecticsm} is a stronger approach to cognitive science
than generic nominalism. This approach posits that cognitive science is a "forum"
from which people from different discplines can discuss or share ideas. There is
a sense, under this approach, that cognitive science involves collaboration between
discplines on some level. This level can be described with the analogy of an interfaith
dialogue: People are tolerant of multiple ideas but there is little to no attempt
to integrate those ideas to form a unified understanding.

Interdisciplinary Ecclecticsm is typically not a very stable approach to cognitive
science. More often than not, it devloves into generic nominalism or evolves into
synoptic integration.

\textit{Related Ideas: Philosophical Ecclecticsm, Syncreticsm, Clinical Pluralism }

\textbf{Synoptic Integration} is the strongest approach to cognitive science. Cognitive
science, under this approach is a deliberate and unique discpline whre knowledge
from cognitive sciences sister disciplines are perfectly integrated.

The word "synoptic" comes from the greek word "sunoptikos" meaning "seeing everything
together". The use of the term "synoptic" truly speaks to the nature of this approach.
The goal is to achiece a coherent understanindg of cognitive phenomena.

Synoptic integration is about making the right connections, and seeing relationships
between disciplines and subjects to study the mind. It is the most widely accepted
approach of cognitve science.

\textit{Related Ideas: SI in Interactive Entertainment, SI in Information Theory, SI in Music Composition}

\section{The Naturalistic Imperative}
Core Ideas: Cognitive Science is influenced by the same naturalism that influences
most of modern day science, Humans have innate desire to learn about the world,
Philosophers are trained in navigating the abstract,

The philosopher aims to both ask the necessary questions and provide a methodology to answer them. (more
on this is in the naturalstic imperative section). This has been the case in multiple
discplines including psychology, physics, mathematics and virtually every modern
scientific discpline.

\subsection{Analysis}
\textbf{Peroid: } Presocratic Thinkers; 469 B.C - 4 B.C \\
\textbf{Key Proponents: } Thales of Miletus \\

% \textbf{Why study them?}
% \begin{itemize}
%   \item They are the intellectual and cultural foundations of western civilization
%   If one wants to understand western civilization one must understand Plato and Aristotle
%   If one is to understand Plato and Aristotle, one would benefit from understanding the pre socratics.
%   \item The Presocratics introduced rational thought to the West. Notions like logos (logic/order)
%   and kosmos also came about during this time period.
%   \item Thales in particular is considered to be the father of philosophy.
% \end{itemize}

\textbf{Main idea (s):}
- The Presocratics ushered in the philosophic and scientific
mindset that would dramatically alter the course of western civilization.
- The concept of rational thought and logos was introduced by presocratic
thinkers.
- Naturalism

"Analysis means to discover those basic processes in terms of which complex mental
phenomena can be comprehensively explained." - JVs Thesis

\textbf{Note: } Pay attention to the way figures of the past thought
rather than the validity of the theory. It often proves more usefel to look at the
rationale behind a conclusion rather than the conclusion itself. This applies especially
to philosophers from Ancient Times, including Thales.

\textbf{Fragments from Thales}
\begin{itemize}
  \item "All is the moist"
    \begin{itemize}
      \item Basic Idea: Thales claims that everything is made of water.
      \item Significance: Thales is the first philosopher to try to answer questions
            about how the world works by appealing to substance rather than mental
            properties / supernatural agency. Thales sought to break down things into
            what they are made of. Him like a lot of other presocratic thinkers
            sought to understand the world in terms of structure.
      \item Why is this profound? Greece was surrounded by water. This theory
            was plausible at the time.
    \end{itemize}
  \item "The lodestone has psyche"
    \begin{itemize}
      \item Basic Idea: Thales observes that magnets are weird.
      \item Significance:
      \item Why is this profound?
    \end{itemize}
  \item "Everything is filled with gods"
    \begin{itemize}
      \item Basic Idea: Things are complex... Keep looking at stuff nothing is boring.
      \item Signifiance: Common Sense explains unfamiliar in terms of the familiar.
      Science explains familiar in terms of the unfamiliar.
      \item Why is this profound?
    \end{itemize}
\end{itemize}

\subsection{Formalisation}
\textbf{Time Period: } Scientific Revolution; 1550 - 1700 (estimated) \\
\textbf{Key Proponents: } Descartes, Copernicus, Newton

\textbf{Main idea (s): }
- This shift from Cosmos to Universe also marked a transformation from an Organic
Worldview to a Mechanical World Picture.
- Take all that wisdom... And convert it into percise terms.
- The origins of the scientific method.
- Descartes invented cartesian graphing. This
- Cartesian graphing
- Science became the process of viewing the familiar in terms of the unfamiliar. Because
of this science may violate our intuition about the world. That does not mean
that intuition does not have a role in science. It does. The point is that intuition
is not necessary condition for the validity of a scientific theory.



\subsection{Mechanisation}
\textbf{Time Period: } Computational Revolution; Early 20th Century \\
\textbf{Key Proponents: } Alan Turing

\textbf{Main idea: } Once you analyzed and formalized you can feed all that logic
into a machine.

\textbf{Turing Test}

\textbf{Questions?}
\begin{enumerate}
  \item What about mechanisation is so key to the naturalistic imperative?
  \item Can you give an example of how mechanisation plays a role in current work
        in cognitive science... other than Artificial Intelligence?
\end{enumerate}

\end{document}
