\documentclass[twoside]{article}
\setlength{\oddsidemargin}{0.25 in}
\setlength{\evensidemargin}{-0.25 in}
\setlength{\topmargin}{-0.6 in}
\setlength{\textwidth}{6.5 in}
\setlength{\textheight}{8.5 in}
\setlength{\headsep}{0.75 in}
\setlength{\parindent}{0 in}
\setlength{\parskip}{0.1 in}

%
% ADD PACKAGES here:
%

\usepackage{amsmath,amsfonts,amssymb,graphicx,mathtools,flexisym}

%
% The following commands set up the lecnum (lecture number)
% counter and make various numbering schemes work relative
% to the lecture number.
%
\newcounter{lecnum}
\renewcommand{\thepage}{\thelecnum-\arabic{page}}
\renewcommand{\thesection}{\thelecnum.\arabic{section}}
\renewcommand{\theequation}{\thelecnum.\arabic{equation}}
\renewcommand{\thefigure}{\thelecnum.\arabic{figure}}
\renewcommand{\thetable}{\thelecnum.\arabic{table}}

%
% The following macro is used to generate the header.
%
\newcommand{\lecture}[4]{
   \pagestyle{myheadings}
   \thispagestyle{plain}
   \newpage
   \setcounter{lecnum}{#1}
   \setcounter{page}{1}
   \noindent
   \begin{center}
   \framebox{
      \vbox{\vspace{2mm}
    \hbox to 6.28in { {\bf COG250H1: Introduction to Cognitive Science
	\hfill Fall 2018} }
       \vspace{4mm}
       \hbox to 6.28in { {\Large \hfill Lecture #1: #2  \hfill} }
       \vspace{2mm}
       \hbox to 6.28in { {\it Lecturer: #3 \hfill Scribes: #4} }
      \vspace{2mm}}
   }
   \end{center}
   \markboth{Lecture #1: #2}{Lecture #1: #2}

   {\bf Note}: {\it LaTeX template courtesy of UC Berkeley EECS dept.}

   {\bf Disclaimer}: {\it These notes have not been subjected to the
   usual scrutiny reserved for formal publications.  They may be distributed
   outside this class only with the permission of the Instructor.}
   \vspace*{4mm}
}
%
% Convention for citations is authors' initials followed by the year.
% For example, to cite a paper by Leighton and Maggs you would type
% \cite{LM89}, and to cite a paper by Strassen you would type \cite{S69}.
% (To avoid bibliography problems, for now we redefine the \cite command.)
% Also commands that create a suitable format for the reference list.
\renewcommand{\cite}[1]{[#1]}
\def\beginrefs{\begin{list}%
        {[\arabic{equation}]}{\usecounter{equation}
         \setlength{\leftmargin}{2.0truecm}\setlength{\labelsep}{0.4truecm}%
         \setlength{\labelwidth}{1.6truecm}}}
\def\endrefs{\end{list}}
\def\bibentry#1{\item[\hbox{[#1]}]}

%Use this command for a figure; it puts a figure in wherever you want it.
%usage: \fig{NUMBER}{SPACE-IN-INCHES}{CAPTION}
\newcommand{\fig}[3]{
			\vspace{#2}
			\begin{center}
			Figure \thelecnum.#1:~#3
			\end{center}
	}
% Use these for theorems, lemmas, proofs, etc.
\newtheorem{theorem}{Theorem}[lecnum]
\newtheorem{lemma}[theorem]{Lemma}
\newtheorem{proposition}[theorem]{Proposition}
\newtheorem{claim}[theorem]{Claim}
\newtheorem{corollary}[theorem]{Corollary}
\newtheorem{definition}[theorem]{Definition}
\newenvironment{proof}{{\bf Proof:}}{\hfill\rule{2mm}{2mm}}

% **** IF YOU WANT TO DEFINE ADDITIONAL MACROS FOR YOURSELF, PUT THEM HERE:

\newcommand\E{\mathbb{E}}

\begin{document}
%FILL IN THE RIGHT INFO.
%\lecture{**LECTURE-NUMBER**}{**DATE**}{**LECTURER**}{**SCRIBE**}
\lecture{2}{Introduction + Categorization (Part 1)}{Anderson Todd}{Ousmane Amadou}
%\footnotetext{These notes are partially based on those of Nigel Mansell.}

% **** YOUR NOTES GO HERE:
\section{An Overview of Cognitive Science (CogSci)}
\subsection{Background / Motivation}
Purpose: Set the stage for providing visions of cognitive sceince, introduce
sub disciplines, briefly describe key events in cognitive science.

\begin{itemize}
  \item Artificial Intelligence
  \item Cognitive Psychology
  \item Philosophy of Mind
  \item Cultural Anthropology
  \item Neuroscience
  \item Semiotics and Linguistics
\end{itemize}

Topic involving cognitive science:
\begin{itemize}
  \item AlphaGo
  \item Computer Vision
  \item High Frequency Trading
  \item Literacy + Grammar
\end{itemize}

In cognitive science, the problem of the mind is being approached from multiple
angles. A \textbf{metaphor} that you will often see in cognitive science is that the brain
is a machine. The fundamental belief is that if science can figure out how to
design and construct a mind, then science can figure out how the mind works. This is
one of the reasons why the study of artificial intellgience and machine learning, is
so intimately connected to the study of the mind. This approach is characterized
by employing \textbf{Design Thinking} to create models of the mind.

There are, however, problems with this approach. Many machine learning systems these
days are very black box. That is, although we understand the basic principles
of how these systems work, they are not fully yet understood.

\subsection{Theoretical Constucts and Plausibility}

A theoretical construct (a.k.a hypothetical construct) is description of
an ideal object (an idea) that is not directly observable. Theoretical constructs
are judged on how accurately they describe the object. A theoretical concept is
strong if and only if it is: multi apt, is supported by converging evidence, is
elegantly formulated.

More Writing Points: Explain Multi Aptnes, Describe the connection between aptness
and metphor

\subsection{What is CogSci?}
We don't know what makes us think.

The one thing we can't account for in our scientific explanations is
how we produce scientific explanations.

CogSci seeks to develop a common language for describing \textbf{cognitive phenomena}
that can be understood through multiple disciplines.

There are three 'visions' for cognitive science:
\begin{enumerate}
  \item Generic Nominalism (Very High Level)
    \begin{itemize}
      \item In its weakest vision, cognitive science is the sum of the following
            fields as the pertain to the study of the mind: psychology, artificial
            intelligence, linguistics, neuroscience, anthropology, and philosophy.
      \item Under this vision the only requirement for doing cognitive science
            is to do something related to the study of the mind. Thus the fields
            mentioned are referred to as the cognitive sciences.
      \item The term generic nominalism has two components. Generic corresponds
            to genre.. Nominalism corresponds to name.
      \item This vision is generally not accepted in third generation cognitive science
    \end{itemize}
  \item Interdisciplinary Ecclecticsm (Still High Level)
  \begin{itemize}
    \item A stonger definition than generic nominalism, not the strongest
    \item Under this vision, different disciplines
    \item This approach is characterized by drawing from CogSci's sub disciplines
          to analyze the mind. Instead of holding to a single paradigm or framework
          of thought, IE seeks to integrate knowledge from all the sub disciplines to
          gain insight into the mind.
    \item Analogy: Interfaith dialgoue. Suppose that buddhists, catholics, muslisms
          and other people fof different faith are invited to a cocktail party.
    \item This model, however, is very unstable. Typically this model either devolves
          into generic nominalism or evolves into synoptic integration.
  \end{itemize}
  \item Synoptic Integration (Lowest level)
  \begin{itemize}
    \item The strongest definition of CogSci.
    \item If you are able to communicate back and forth between the involved
          cognitive science disciplines, you reach perfect integration.
    \item CogSci, under this model, is a unique discipline. Doing CogSci is deliberate.
  \end{itemize}
\end{enumerate}

\section{Naturalistic Imperative}
Core Ideas: Humans have innate desire to learn about natural world, Philosophers
are trained thinkers and are skilled in navigating the abstract,

The Naturalistic Imperative is a term coined by John Vervaeke. One of the goals
of science is to 'naturalize' our understanding of the universe. There is an innate
human desire to bring our minds in line with all the scientific disciplines.

It comprises of three parts. Analysis, Formalization, and Mechanization. To understand
the naturalistic imperative it is useful to look at previous scientific revolutions:

Philosophers are trained in the abstract.

\subsection{Analyze}
Timeline
1. Presocratics (Thales of Miletus)
2. Socrates, Plato and Aristotle
3. The Enlightenment

\subsubsection{The Presocratics (469 B.C - 4 B.C)}


\href{https://www.youtube.com/watch?v=ZkMAx04jDx0}{Video A}

\textbf{Main idea:} The Presocratics ushered in the philosophic and scientific
mindset that would dramatically alter the course of western civilization.
The concept of rational thought and logos was originated by presocratic thinkers.

In general, Ancient greek philosophers form the intellectual and cultural foundations
of western civilization. If one wants to understand western civilization one must understand
the works of Socrates, Plato and Aristotle.


Talking points:
Introduction to the Presocratics:
(1) Presocratic philosophers
  (0) Why study them?

    (i) They are the intellectual and cultural foundations of western civilization
        If one wants to understand western civilization one must understand Plato
        and Aristotle.. If one is to understand Plato and Aristotle, one would
        benefit from understanding the pre socratics.
    (ii) Presocratics birthed rational thought... It is here that the logos
         came about... The fact that rational thought came from here
  (1) Not actually philosophers, wouldn't call themselves philosophers either..
      A better way to describe these group of ople would be thinkers
  (2) End of Presocratic era: The nature of philosophy changed from studying
  nature to .. Socrates stopped focusing on metaphysical questions, becuase
  he felt that studying those questions had no impact on his life. Instead, he started
  turned his attention to practical morality and political thought.
(2) Thales


Thales
Not much left on him
We know three things

\subsection{Formalize}

\subsection{Mechanize}

 \end{document}
