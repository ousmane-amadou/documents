\documentclass[twoside]{article}
\setlength{\oddsidemargin}{0.25 in}
\setlength{\evensidemargin}{-0.25 in}
\setlength{\topmargin}{-0.6 in}
\setlength{\textwidth}{6.5 in}
\setlength{\textheight}{8.5 in}
\setlength{\headsep}{0.75 in}
\setlength{\parindent}{0 in}
\setlength{\parskip}{0.1 in}

%
% ADD PACKAGES here:
%

\usepackage{amsmath,amsfonts,amssymb,graphicx,mathtools,flexisym}

%
% The following commands set up the lecnum (lecture number)
% counter and make various numbering schemes work relative
% to the lecture number.
%
\newcounter{lecnum}
\renewcommand{\thepage}{\thelecnum-\arabic{page}}
\renewcommand{\thesection}{\thelecnum.\arabic{section}}
\renewcommand{\theequation}{\thelecnum.\arabic{equation}}
\renewcommand{\thefigure}{\thelecnum.\arabic{figure}}
\renewcommand{\thetable}{\thelecnum.\arabic{table}}

%
% The following macro is used to generate the header.
%
\newcommand{\lecture}[4]{
   \pagestyle{myheadings}
   \thispagestyle{plain}
   \newpage
   \setcounter{lecnum}{#1}
   \setcounter{page}{1}
   \noindent
   \begin{center}
   \framebox{
      \vbox{\vspace{2mm}
    \hbox to 6.28in { {\bf COG250H1: Introduction to Cognitive Science
	\hfill Fall 2018} }
       \vspace{4mm}
       \hbox to 6.28in { {\Large \hfill Lecture #1: #2  \hfill} }
       \vspace{2mm}
       \hbox to 6.28in { {\it Lecturer: #3 \hfill Scribes: #4} }
      \vspace{2mm}}
   }
   \end{center}
   \markboth{Lecture #1: #2}{Lecture #1: #2}

   {\bf Note}: {\it LaTeX template courtesy of UC Berkeley EECS dept.}

   {\bf Disclaimer}: {\it These notes have not been subjected to the
   usual scrutiny reserved for formal publications.  They may be distributed
   outside this class only with the permission of the Instructor.}
   \vspace*{4mm}
}
%
% Convention for citations is authors' initials followed by the year.
% For example, to cite a paper by Leighton and Maggs you would type
% \cite{LM89}, and to cite a paper by Strassen you would type \cite{S69}.
% (To avoid bibliography problems, for now we redefine the \cite command.)
% Also commands that create a suitable format for the reference list.
\renewcommand{\cite}[1]{[#1]}
\def\beginrefs{\begin{list}%
        {[\arabic{equation}]}{\usecounter{equation}
         \setlength{\leftmargin}{2.0truecm}\setlength{\labelsep}{0.4truecm}%
         \setlength{\labelwidth}{1.6truecm}}}
\def\endrefs{\end{list}}
\def\bibentry#1{\item[\hbox{[#1]}]}

%Use this command for a figure; it puts a figure in wherever you want it.
%usage: \fig{NUMBER}{SPACE-IN-INCHES}{CAPTION}
\newcommand{\fig}[3]{
			\vspace{#2}
			\begin{center}
			Figure \thelecnum.#1:~#3
			\end{center}
	}
% Use these for theorems, lemmas, proofs, etc.
\newtheorem{theorem}{Theorem}[lecnum]
\newtheorem{lemma}[theorem]{Lemma}
\newtheorem{proposition}[theorem]{Proposition}
\newtheorem{claim}[theorem]{Claim}
\newtheorem{corollary}[theorem]{Corollary}
\newtheorem{definition}[theorem]{Definition}
\newenvironment{proof}{{\bf Proof:}}{\hfill\rule{2mm}{2mm}}

% **** IF YOU WANT TO DEFINE ADDITIONAL MACROS FOR YOURSELF, PUT THEM HERE:

\newcommand\E{\mathbb{E}}

\begin{document}
%FILL IN THE RIGHT INFO.
%\lecture{**LECTURE-NUMBER**}{**DATE**}{**LECTURER**}{**SCRIBE**}
\lecture{2}{Introduction + Categorization I (Resemblance Theory)}{Anderson Todd}{Ousmane Amadou}
%\footnotetext{These notes are partially based on those of Nigel Mansell.}

% **** YOUR NOTES GO HERE:
\section{An Overview of Cognitive Science (CogSci)}
\subsection{Background / Motivation}
Purpose: Set the stage for providing visions of cognitive sceince, introduce
sub disciplines, briefly describe key events in cognitive science.

\begin{itemize}
  \item Artificial Intelligence
  \item Cognitive Psychology
  \item Philosophy of Mind
  \item Cultural Anthropology
  \item Neuroscience
  \item Semiotics and Linguistics
\end{itemize}

Topic involving cognitive science:
\begin{itemize}
  \item AlphaGo
  \item Computer Vision
  \item High Frequency Trading
  \item Literacy + Grammar
\end{itemize}

In cognitive science the problem of the mind is being approached from multiple
angles. One of these approaches is characterized by employing \textbf{design thinking}
to create models of the mind. The brain as a machine is a \textbf{metaphor} that
you will often see in cognitive science. The fundamental belief is that if science
can figure out how to design and construct a mind, then science can figure out how the mind works.
This is one of the reasons why the study of artificial intellgience is so intimately
connected to the study of the mind.

There are, however, problems with this approach. Many machine learning systems these
days are very black box. That is, although we understand the basic principles
of how these systems work, they are not fully yet understood.

\subsection{Theoretical Constucts and Plausibility}

A theoretical construct (a.k.a hypothetical construct) is description of
an ideal object (an idea) that is not directly observable. Theoretical constructs
are judged on how accurately they describe the object.

Streams of evidence....
Converging input and elegant output -> Plausibility

You want your converging input and elegant
If your plausibility is deep then you may have a theory that is profound. This is
the goal of integrating multiple disciplines.

Example: Attachment Theory and Nueroscience in relation to trust in relationships...
What does trust mean in mathematical terms.

A theoretical construct is strong if and only if it is: multi apt, is supported by converging evidence, is
elegantly formulated.

More Writing Points: Explain Multi Aptnes, Describe the connection between aptness
and metaphor

\subsection{Metaphor}
\begin{itemize}
  \item Etymology: From the latin word metaphora meaning "carry over", which in turn
  came from the greek word metaphori meaning "transfer"
  \item Signifiance: Metaphor provides a structural backbone for knowledge transfer.
  \item Example: "Sam is such a pig!". There is something about the aptness
  of this particular metaphor.
\end{itemize}

\subsection{What is CogSci?}
CogSci seeks to develop a common language for describing \textbf{cognitive phenomena}
that can be understood through multiple disciplines. However, the core of cognitive
science is considered to be philosophy.

Ironically, one thing science can't account for in scientific explanations is how
humans produce scientific explanations. Because of this there is a sense in which we are deeply
alienated from the natural world. That is, there is a \textbf{u-shaped hole} in
our understanding of it. (notice that u-shaped hole is a metaphor) One of the goals
of cognitive science to fill this gap, and complete our understanding of the natural world.

There are three 'visions' for cognitive science:
\begin{enumerate}
  \item Generic Nominalism (Very High Level)
    \begin{itemize}
      \item In its weakest vision, cognitive science is the sum of the following
            fields as the pertain to the study of the mind: psychology, artificial
            intelligence, linguistics, neuroscience, anthropology, and philosophy.
      \item Under this vision the only requirement for doing cognitive science
            is to do something related to the study of the mind. Thus the fields
            mentioned are referred to as the cognitive sciences.
      \item The term generic nominalism has two components. Generic corresponds
            to genre.. Nominalism corresponds to name.
      \item This vision is generally not accepted in third generation cognitive science
    \end{itemize}
  \item Interdisciplinary Ecclecticsm (Still High Level)
  \begin{itemize}
    \item A stonger definition than generic nominalism, not the strongest
    \item Under this vision, different disciplines
    \item This approach is characterized by drawing from CogSci's sub disciplines
          to analyze the mind. Instead of holding to a single paradigm or framework
          of thought, IE seeks to integrate knowledge from all the sub disciplines to
          gain insight into the mind.
    \item Analogy: Interfaith dialgoue. Suppose that buddhists, catholics, muslisms
          and other people fof different faith are invited to a cocktail party.
    \item This model, however, is very unstable. Typically this model either devolves
          into generic nominalism or evolves into synoptic integration.
  \end{itemize}
  \item Synoptic Integration (Lowest level)
  \begin{itemize}
    \item The strongest definition of CogSci.
    \item If you are able to communicate back and forth between the involved
          cognitive science disciplines, you reach perfect integration.
    \item CogSci, under this model, is a unique discipline. Doing CogSci is deliberate.
  \end{itemize}
\end{enumerate}

\section{Naturalistic Imperative}
Core Ideas: Humans have innate desire to learn about natural world, Philosophers
are trained in navigating the abstract, The Naturalistic Imperative is in a
sense part of the philosophical enteprise as well

The Naturalistic Imperative is a term coined by John Vervaeke. It is a term used
to describe sciences' goal of 'natrualizing' our understanding of the
universe. It comprises of three parts. Analysis, Formalization, and Mechanization.

To understand the naturalistic imperative it is useful to look at previous scientific revolutions.

Philosophers are trained in the abstract problems of trying to work out
language and questions to a really high degree of percision. The philosopher aims
to both ask the necessary questions and provide a methodology to answer them. (more
on this is in the naturalstic imperative section) This has been the case in multiple
discplines including psychology, physics, mathematics and virtually every modern
scientific discpline.

\subsection{Analyze}
What does it mean to analyze? To break down concepts in its most fundamental
parts.

\subsubsection{The Presocratics (469 B.C - 4 B.C)}
\textbf{Main idea:} The Presocratics ushered in the philosophic and scientific
mindset that would dramatically alter the course of western civilization. The concept
of rational thought and logos was introduced by presocratic thinkers.

\textbf{Key Proponents: } Thales of Miletus

\textbf{Note:} Pay more attention to the WAY thinkers of the past thought
rather than the validity of the theory. It is more useful to understand HOW a past
thinker reaches a conclusion rather than the conclusion itself. This applies especially
to philosophers from Ancient Times, including Thales.

\textbf{Why study them?}
\begin{itemize}
  \item  They are the intellectual and cultural foundations of western civilization
  If one wants to understand western civilization one must understand Plato and Aristotle
  If one is to understand Plato and Aristotle, one would benefit from understanding the pre socratics.
  \item  The Presocratics introduced rational thought to the West. Notions like logos (logic/order)
  and kosmos also came about during this time period.
  \item Thales in particular is considered to be the father of philosophy.
\end{itemize}

Fragments from Thales
\begin{itemize}
  \item "All is the moist"
    \begin{itemize}
      \item Basic Idea: Thales claims that everything is made of water.
      \item Signifiance: Thales is the first philosophers to try to answer questions
            about how the world works by appealing to substance rather than mental
            properties / supernatural agency. Thales sought to break down things into
            what they are made of. Him like a lot of other presocratic thinkers
            sought to understand the world in terms of structure.
      \item Why is this profound? Greece was surrounded by water. This theory
            was plausible at the time.
    \end{itemize}
  \item "The lodestone has psyche"
    \begin{itemize}
      \item Basic Idea: Thales observes tha magnets are weird.
      \item Signifiance:
      \item Why is this profound?
    \end{itemize}
  \item "Everything is filled with gods"
    \begin{itemize}
      \item Basic Idea: Things are complex... Keep looking at stuff nothing is boring.
      \item Signifiance: Common Sense explains unfamiliar in terms of the familiar.
      Science explains familiar in terms of the unfamiliar.
      \item Why is this profound?
    \end{itemize}
\end{itemize}


\subsection{Formalize}

\subsection{Mechanize}

\end{document}
