\documentclass[twoside]{article}
\setlength{\oddsidemargin}{0.25 in}
\setlength{\evensidemargin}{-0.25 in}
\setlength{\topmargin}{-0.6 in}
\setlength{\textwidth}{6.5 in}
\setlength{\textheight}{8.5 in}
\setlength{\headsep}{0.75 in}
\setlength{\parindent}{0 in}
\setlength{\parskip}{0.1 in}

%
% ADD PACKAGES here:
%

\usepackage{amsmath,amsfonts,amssymb,graphicx,mathtools,flexisym}

%
% The following commands set up the lecnum (lecture number)
% counter and make various numbering schemes work relative
% to the lecture number.
%
\newcounter{lecnum}
\renewcommand{\thepage}{\thelecnum-\arabic{page}}
\renewcommand{\thesection}{\thelecnum.\arabic{section}}
\renewcommand{\theequation}{\thelecnum.\arabic{equation}}
\renewcommand{\thefigure}{\thelecnum.\arabic{figure}}
\renewcommand{\thetable}{\thelecnum.\arabic{table}}

%
% The following macro is used to generate the header.
%
\newcommand{\lecture}[4]{
   \pagestyle{myheadings}
   \thispagestyle{plain}
   \newpage
   \setcounter{lecnum}{#1}
   \setcounter{page}{1}
   \noindent
   \begin{center}
   \framebox{
      \vbox{\vspace{2mm}
    \hbox to 6.28in { {\bf COG250H1: Introduction to Cognitive Science
	\hfill Fall 2018} }
       \vspace{4mm}
       \hbox to 6.28in { {\Large \hfill Lecture #1: #2  \hfill} }
       \vspace{2mm}
       \hbox to 6.28in { {\it Lecturer: #3 \hfill Scribes: #4} }
      \vspace{2mm}}
   }
   \end{center}
   \markboth{Lecture #1: #2}{Lecture #1: #2}

   {\bf Note}: {\it LaTeX template courtesy of UC Berkeley EECS dept.}

   {\bf Disclaimer}: {\it These notes have not been subjected to the
   usual scrutiny reserved for formal publications.  They may be distributed
   outside this class only with the permission of the Instructor.}
   \vspace*{4mm}
}
%
% Convention for citations is authors' initials followed by the year.
% For example, to cite a paper by Leighton and Maggs you would type
% \cite{LM89}, and to cite a paper by Strassen you would type \cite{S69}.
% (To avoid bibliography problems, for now we redefine the \cite command.)
% Also commands that create a suitable format for the reference list.
\renewcommand{\cite}[1]{[#1]}
\def\beginrefs{\begin{list}%
        {[\arabic{equation}]}{\usecounter{equation}
         \setlength{\leftmargin}{2.0truecm}\setlength{\labelsep}{0.4truecm}%
         \setlength{\labelwidth}{1.6truecm}}}
\def\endrefs{\end{list}}
\def\bibentry#1{\item[\hbox{[#1]}]}

%Use this command for a figure; it puts a figure in wherever you want it.
%usage: \fig{NUMBER}{SPACE-IN-INCHES}{CAPTION}
\newcommand{\fig}[3]{
			\vspace{#2}
			\begin{center}
			Figure \thelecnum.#1:~#3
			\end{center}
	}
% Use these for theorems, lemmas, proofs, etc.
\newtheorem{theorem}{Theorem}[lecnum]
\newtheorem{lemma}[theorem]{Lemma}
\newtheorem{proposition}[theorem]{Proposition}
\newtheorem{claim}[theorem]{Claim}
\newtheorem{corollary}[theorem]{Corollary}
\newtheorem{definition}[theorem]{Definition}
\newenvironment{proof}{{\bf Proof:}}{\hfill\rule{2mm}{2mm}}

% **** IF YOU WANT TO DEFINE ADDITIONAL MACROS FOR YOURSELF, PUT THEM HERE:

\newcommand\E{\mathbb{E}}

\begin{document}
%FILL IN THE RIGHT INFO.
%\lecture{**LECTURE-NUMBER**}{**DATE**}{**LECTURER**}{**SCRIBE**}
\lecture{3}{Categorization II (Classical Theory, and Prototype Theory)}{Anderson Todd}{Ousmane Amadou}
%\footnotetext{These notes are partially based on those of Nigel Mansell.}

\section{section name}

Science is the process of viewing the familiar in terms of the unfamiliar. Because
of this science may violate our intuition about the world. That does not mean
that intuition does not have a role in science. It does. The point is that intuition
is not necessary condition for the validity of a scientific theory.

What is a category?
- Coding of Experience: Means we don't have to classify everything as a radical
individual

A group of things that belong together.

\textbf{Key Idea: } Everytime you hear "All x are y" be very critical of it.}

You can treat everything in your environment as an instance of a type. However,
there is problems with this

However, things have an infinte amount of feature complexity that can form categories
in our world.

Categorical Problems: Uniqueness of a chair violating your categorical
assumption


Main Ideas:
- Tversky's formula for similarity
- Salience

Some people play with the idea that categories are innate and that humans
discover them as they mature.

1. Salience

% **** YOUR NOTES GO HERE:
\section{Categorization}
Naive model - We look at things and decide based off similarity whether
things belong in the same category. One of these things belong together
and on of these things do not. Rely on your perception and judgement to
form categories.

This naive model (common sense assumption) holds up at first glance. One of the
problems with this is that (as nelson goodman points out) if we are treating
similarity as a question of partial identity of terms, then things have an
infinite amount of similarity of dissimilarity. This is not very plausible
since humans have a limited amount of cognitive power.

\subsection{The Resemblance Theory}
Name coined by Lance Ribs

Family of resemblance...

Family analogy

defenses:
well ok. Every member of a cateogry are equally representative of the cateogry.
criticsms:
if similarity judgments are supposed to drive similarity judgements
Purpose: Set the stage for providing visions of cognitive sceince, introduce
sub disciplines, briefly describe key events in cognitive science.


\subsection{Geometric Notion - Tversky's Formula}
$$ Sim(I, J) = af(I, J) + bf(I, J) + cf(I, J) $$
where:

f is a function that measures the salience of each set of features.

a, b,  and  c are  parameters  that  determine the relative contribution
of the three feature sets.

Problems: Falls prey to homuncular fallacy

\subsection{Classical Theory}
Motivation:
Smith: Categorization is running on some feature of the world. Categorization
is in some sense a bottom up process.

Response: Categorization is rather a top down process. It is more plausible that
we use concepts to categorize objects. We are imposing a concept into cateogries.


Aristotle championed the classical theory of concepts.

Somehow we are using concepts to do categorization.

\textbf{Theory: }

\textbf{Evidence: }

\textbf{Criticsms: }
1. Wittgenstein on the game

\subsection{Prototpye Theory}

\textbf{Theory: }

\textbf{Evidence: }

\textbf{Criticsms: }


\end{document}
